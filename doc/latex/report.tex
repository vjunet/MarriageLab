\documentclass[11pt]{article}
\usepackage{geometry}                
\geometry{letterpaper}                   

\usepackage{graphicx}
\usepackage{amssymb}
\usepackage{epstopdf}
\usepackage{natbib}
\usepackage{amssymb, amsmath}

\usepackage[autolinebreaks,numbered]{mcode} % matlab code listings

\usepackage{natbib} % bibliography

\usepackage{pdfpages}

\usepackage[hidelinks]{hyperref}

\usepackage{caption}
\usepackage{subcaption}

\usepackage{enumitem}

\DeclareGraphicsRule{.tif}{png}{.png}{`convert #1 `dirname #1`/`basename #1 .tif`.png}

\setlength{\parskip}{0.1cm}			% distance between paragraphs
%\renewcommand{\baselinestretch}{1.1}	% control line spacing in text

\DeclareCaptionFont{black}{ \color{black} }
\DeclareCaptionFormat{listing}{
  \parbox{\textwidth}{\hspace{15pt}#1#2#3}
}
\captionsetup[lstlisting]{ format=listing, labelfont=black, textfont=black, singlelinecheck=false, margin=0pt, font=normalsize }

\newcommand{\comment}[1]{}

%\title{Title}
%\author{Name 1, Name 2}
%\date{date} 

\begin{document}



\thispagestyle{empty}

\begin{center}
\includegraphics[width=5cm]{ETHlogo.eps}

\bigskip


\bigskip


\bigskip


\LARGE{ 	Lecture with Computer Exercises:\\ }
\LARGE{ Modelling and Simulating Social Systems with MATLAB\\}

\bigskip

\bigskip

\small{Project Report}\\

\bigskip

\bigskip

\bigskip

\bigskip


\begin{tabular}{|c|}
\hline
\\
\textbf{\LARGE{Stable Marriage Problem}}\\
%\textbf{\LARGE{...}}\\
\\
\hline
\end{tabular}
\bigskip

\bigskip

\bigskip

\LARGE{Valentin Junet \& Samuel Imfeld}



\bigskip

\bigskip

\bigskip

\bigskip

\bigskip

\bigskip

\bigskip

\bigskip

Zurich\\
December 2014\\

\end{center}



\newpage

%%%%%%%%%%%%%%%%%%%%%%%%%%%%%%%%%%%%%%%%%%%%%%%%%

\newpage

\setcounter{page}{1}
\pagenumbering{roman}

\comment{
\section*{Agreement for free-download}
\bigskip


\bigskip


\large We hereby agree to make our source code for this project freely available for download from the web pages of the SOMS chair. Furthermore, we assure that all source code is written by ourselves and is not violating any copyright restrictions.

\begin{center}

\bigskip


\bigskip


\begin{tabular}{@{}p{3.3cm}@{}p{6cm}@{}@{}p{6cm}@{}}
\begin{minipage}{3cm}

\end{minipage}
&
\begin{minipage}{6cm}
\vspace{2mm} \large Valentin Junet

 \vspace{\baselineskip}

\end{minipage}
&
\begin{minipage}{6cm}

\large Samuel Imfeld

\end{minipage}
\end{tabular}


\end{center}
}
\includepdf{page_i}
\newpage

%%%%%%%%%%%%%%%%%%%%%%%%%%%%%%%%%%%%%%%

%\includepdf{declaration-originality}
\includepdf{page_ii}

% IMPORTANT
% you MUST include the ETH declaration of originality here; it is available for download on the course website or at http://www.ethz.ch/faculty/exams/plagiarism/index_EN; it can be printed as pdf and should be filled out in handwriting


%%%%%%%%%% Table of content %%%%%%%%%%%%%%%%%

\newpage

\tableofcontents

\newpage

\setcounter{page}{1} % start page numbering here
\pagenumbering{arabic}

%%%%%%%%%%%%%%%%%%%%%%%%%%%%%%%%%%%%%%%



\section{Abstract}

Online dating or match-making websites are flourishing these days. More and more people rely on their algorithms when
searching for their Mr. Right, or Mrs. Right, respectively. Algorithms for match-making are therefore of quite some interest.

An important result in this concern is the so-called Gale Shapley algorithm proposed by  \citet{1962}. In 2012, Shapley even received the Nobel
Prize in Economics for his work. The goal of this paper is to discuss the original model described by \citet{1962} and advance the model in some
sense. Namely we are going to introduce two changes to the model: The incompleteness of information and the possibility of preference changes.
The modifications will be discussed further in section \ref{mods}.

It is however not our claim that these changes applied to the model will make it an exact description of reality. Our goal is to study
the repercussions on the stability and other significant indicators that show up when applying the modifications.

%TODO research question

\comment{
\begin{enumerate}
  \item In the original model every node knows all the other nodes of the opposite gender. In a setting like a database of some
  match-making site this may be true. But as soon as the number of nodes gets big, it costs a huge amount of computation time to
  consider all nodes. In reality, information about the nodes of opposite is never complete (this would mean knowing about 3.5 billion people).
  \item It is also conceivable that at some point a node might change its opinion about other nodes and rearrange them in his preference rating
\end{enumerate}

}

\section{Individual contributions} %ok

Code was written by both. In the report, Valentin focused on the chapters 3, 4 and 6.1 whereas Samuel concentrated on chapters 1, 5 and 6.2.

\section{Introduction and Motivations}

The stable marriage problem is quite well known; it is originally an interesting mathematical 
problem, but it also has something catchy: Everyone has his own idea of what stability is, how we 
think it is optimal, and how we would like to be matched, or how we think it should be done. Then there comes this emotionless algorithm
that pretends -- with the support of some definitions and mathematical proofs -- to find all of this for us and to do it as least as good
as we would. Of course this is a little exaggerated. But when we think about those internet sites that match people according to some  criteria,
is't it what we're doing? Letting some numbers find an objective answer to those questions, to our subjective questions?

In order to prevent misunderstandings one must very precisely answer the following questions: What is stability? How can a stable situation be optimal?
Of course answering those questions would result in a rather philosophical discussion which is not the goal of our paper.
We will only be considering some very abstract and theoretical nodes which, for some practical and rhetorical reasons, will be called men or women.
Nevertheless stability and other related indicators (which we’re going to introduce soon enough) will be the center of our attention.

The key points of our work presented here are to generate random links (or ''friendship'') between men and women (which in the following 
discussions will be referred to by nodes), to classify those links in a random way in order to get a preference list and to apply the algorithm 
of Gale and Shapley which matches the nodes according to their preferences. Then we modify some parameters and interpret the results. For a more precise 
description of the procedure, you may refer to the next section.

\section{Description of the Model}

\subsection{Gale and Shapley's algorithm}

Let's first describe the algorithm of Gale and Shapley. The setting is of a very abstract nature. There is a certain number of nodes, half of
them are men and the other half are women. The first step is that every node makes a preference list, i. e. a ranking of all the nodes of oppposite
sex. Then the algorithm proceeds as follows:
\begin{enumerate}
  \item single men propose to their favorite women
  \item
  \begin{itemize}
    \item women accept their favorite suitor or their only suitor
    \item if a woman has a fiance and a suitor, she chooses the one she likes better
    \item if a woman rejects a man, then the next woman on the rejected man's list is his new favorite
  \end{itemize}
  \item if the single men still have women on their list of preferences, go to step 1
\end{enumerate}
So the algorithm terminates when all nodes have a partner. This is obviously always achieved if the length of all the preference lists
corresponds to the number of nodes of opposite sex (see \citet[p. 14, Theorem 1]{1962}).
Gale and Shapley also proved that this algorithm always terminates with an optimal stability. For this purpose they gave precise definitions
of stability and optimality:
\\
\\
\textbf{Definition 1: }\textit{An assignment of applicants to colleges will be called unstable if there are two applicants
 $\alpha$ and $\beta$ who are assigned to colleges A and B, respectively, although $\beta$ prefers A to B and A prefers $\beta$
  to $\alpha$.\footnote{\citet[p. 10]{1962}}}
\label{eq:stable}
\\
\\
\textbf{Definition 2: }\textit{A stable assignment is called optimal if every applicant is at
least as well off under it as under any other assignment.\footnote{\citet[p. 10]{1962}}}
\label{eq:optimal}
\\
\\
These definitions are in terms of colleges and applicants because in their paper they first start with this setting and then go on to the
special case of stable marriages. Translated to the words of match-making the definitions would be something like: ''A set of marriages is called
unstable if under it there are a man and a woman who are not married to each other but prefer each other to their actual mates''\footnote{\citet[p. 11]{1962}}
and ''A stable assignment is called optimal if every man is at least as well off under it as under any other assignment''. 

\subsection{Modifications} \label{mods}

We will now present in more detail the modifications we made to this algorithm. Our purpose was 
to modify and adapt it so that it represents more accurately the reality. Of course it is fundamentally 
impossible to make a perfect representation of reality through this: If for example a man 
goes into a bar, he doesn't particularly make a list of the women he prefers and propose to each 
woman according to his list (or at least this is not systematically observed). 
However, there are some modifications we can make to make it more realistic. One of these is that 
a man doesn't necessarily know every women (and vice versa). To improve this point, we modified 
the list of preferences so that men and women only know each other in a restricted area. For 
example, a man who lives in China doesn't know a woman who lives in Switzerland, so he won't 
propose to her (we didn't consider the case where the man in China could know the Swiss 
woman though internet or by traveling, he only knows the women ''around'' him). We also considered 
the case in which a man knows a woman in his area but she doesn't know him and then this man can 
propose to her and she might accept or not. 
The other modification we made is that people, obviously, can change their minds. We modified it so 
that during the process either only the men, only the women or both can change their mind. These 
changes of preferences are made each ''round'' (each time a single man proposes to a woman) with a 
certain probability which is fixed.

We also introduced some new concepts, namely the number of ''dumps'' and the ''optimality index''. The 
number of ''dumps'' is the number of times a woman rejected her fiance for another man. This can 
be seen has an indication of time, the more people ''get dumped'', the longer it takes to reach a final 
assignment. 
The optimality index is defined as the sum over all ranks of the actual partners
in the preference list of their respective partners, divided by two times the number of men/women squared (resulting in a normalization, e. g. 
it only takes values between zero and one). This should somehow give information about the optimality of the outcome, although it does
not correspond to definition 2 in section \ref{eq:optimal}. A low optimality index corresponds to a rather optimal assignment whereas a high
optimality index indicates a not so optimal assignment.

\section{Implementation}

\subsection{Conventions}

The nodes (men resp. women) are referenced by integers from one to n where n is the input
size (number of men/women).

Accordingly, a preference list is a permutation of the first n integers: The element $\sigma(1)$ 
is the most preferred node, $\sigma(2)$ the second most preferred and so on. In the case of 
non-complete information, there are $m<n$ distinct integers followed by zeros to complete the 
sequence.

\subsection{Generating the Preference Lists}

\subsubsection{Random}

For testing purposes we first coded a generator for random preference matrices which 
basically calls \mcode{randperm(n)} n times, resulting in a matrix whose rows represent the 
preference list of a single node. The generated matrix can be used to simulate the classic 
case with complete information.

\subsubsection{The Plane Algorithm}

The question that now arises is the following: How can I generate a preference matrix
with non-complete information that represents reality in an appropriate manner? In 
particular there should be a mechanism for controlling the number of non-zero entries 
in the preference lists (i. e. controlling the degree of completeness of information). The 
number of non-zero elements should, however, not be the same for each node (this would 
mean that each node knows the same fixed number of people, which is clearly not a good 
representation of real situations). Additionally, situations like 'man x knows woman y but 
woman y does not know man x' should be considered.
To achieve these characteristics we developed the 'generatePlane' algorithm: It is based on 
the idea that a node only knows his neighbours (e. g. the people in his town or community). 
Therefore his preference list should consist only of nodes that are closer than a certain 
distance, which we will call the visibility radius in the further discussions. We realized this 
by assigning to each node a random position in a two dimensional plane. For simplicity they 
are distributed in $[0,1]\times[0,1]$. 
\\
\\
\textbf{Definition 3: }\textit{The visibility radius defines the neighbourhood of a node in $[0,1]\times[0,1]$. A node can only know other
nodes that have a distance smaller than the visibility radius.\label{eq:3}}
\\
\\
To avoid border effects, the edges are connected (one could 
view it as a torus). For the actual generation of the preference lists one just has to iterate 
through all nodes and determine all nodes of opposite sex that are in the disc of visibility 
radius. We then chose to use a random permutation of these neighbourhood nodes to keep it 
simple.

A constant visibility radius implies in particular that if node x knows node y, then also node y knows node x. But as we don't want this
to be the case all the time we also made an option for a random visibility radius that is updated every step.

\subsection{Making Matches}

This is the part where the algorithm proposed by \citet{1962} actually comes into play. An implementation in MATLAB is shown in listing
\ref{lst:match} (The basic ideas for the implementation are taken from \citet{rosetta} and adapted to MATLAB).

\lstinputlisting[ captionpos=b, caption=makeMatch code skeleton, label=lst:match]{listing_makeMatch.m}

The main structures are the arrays \mcode{freemen} and \mcode{engaged}. The array \mcode{freemen} contains all free men (first column: indices of
men, second column: 1 for free, 0 for not free) and \mcode{engaged} contains the engagements produced by the algorithm (first column: indices of men, 
second column: indices of women). The main loop starts in line 6 and continues until all men are in an engagement. Then in the loop
the first free man is picked (line 7) and his most preferred girl is determined (line 8). Now one has to distinguish between two cases:

\begin{itemize}
  \item The girl has no engagement: Everything is fine, the man and the girl are now engaged (lines 11 and 12). Also the man is not free
  anymore (line 13).
  \item The girl alread has a fiance: In this case one has to do another distinction:
  \begin{itemize}
    \item The girl prefers the new man to her fiance: A new engagement is made and the old one is cancelled (lines 18-21).
    One also has to update \mcode{freemen}.
    \item The engagements remain unchanged, but the girl is removed from the man's preference list because she is not attainable for him.
  \end{itemize}
\end{itemize}

In the end the engagements are checked with the \mcode{checkEngagements} algorithm described in section \ref{check}. However, when applying the
modifications described in section \ref{mods} and still using this algorithm one runs into problems (as expected). Therefore we had to adapt
the \mcode{makeMatch} algorithm to be able to handle the following situations:

\begin{itemize}
  \item An unknown man proposes to a woman who is not engaged: We decided that in this case the proposing man should have a chance to succeed
  with his proposal, but this should not always be the case. Therefore we implemented a random decision with a certain probability for accepting
  (typically around 0.25). This can be seen as a simulation of the real-life situation 'the woman gets to know the unknown man
  and gets to like him (or not)'.
  \item An unknown node proposes to a woman who is engaged: We decided to apply the same procedure as above. One could argue that the probability
  for accepting should be lower because she already has a fiance, but we left that out for the sake of simplicity.
  \item The preference list of a man is empty but he is not engaged: In this case the man is just left with no partner.
\end{itemize}

The implementation of the above points will not be discussed here any further. The final algorithm is in the appendix for reference.

Finally we added the preference change functionality: In each iteration step a random decision is made between changing preferences or not,
using the probability given in the additional parameter \mcode{changerate}. If the answer is positive, the preference list of one randomly chosen node
is perturbed a little: A random node in the preference list is chosen and switched with the node one rank lower (for example \mcode{[4,2,3,1]} 
becomes \mcode{[4,3,2,1]} after switching nodes 2 and 3). There is also a parameter to determine whether only men, only women or both should
change their preferences.

\subsection{Checking the Engagements} \label{check}

An important indicator for the later discussion is the stability of the engagements. It can be checked using this algorithm. The main loop is
shown in listing \ref{lst:check} (again the basic structure is inspired by \citet{rosetta})

\lstinputlisting[ captionpos=b, caption=checkEngagements code fragment, label=lst:check]{listing_checkEngagements.m}

The main loop is an iteration over all men (line 1), but only those who are engaged because having no partner is not considered as an instability
(lines 3-5). After having retrieved all the indices and preference lists, one iterates over all nodes that appear before the actual partner in their
respective preference lists and checks whether there is an instability (see definition 1 in section \ref{eq:stable}). This results in two for loops, one for the
man and one for the woman he is engaged to.

\subsection{Simulation}

The actual simulation makes iterated calls to \mcode{generatePlane} and \mcode{makeMatch} to simulate the modified Gale-Shapley Algorithm
for different parameter settings. The input parameters are:

\begin{itemize}
  \item \mcode{n}: the input size $n\in\mathbb{N}$
  \item \mcode{mode, radius}: mode determines the choice of the visibility radius (either random or constant), and if the radius is constant
  then it can be set with the parameter radius $r\in[0,0.5]$
  \item \mcode{changerate}: the rate at which preference changes are performed $c\in[0,1]$
  \item \mcode{p}: determines whose preferences are changed $p\in\left\{ {0, 0.5, 1}\right\}$
\end{itemize}

\noindent The resulting output variables are:

\begin{itemize}
  \item no. of instabilities: number of instabilities according to definition 1 in section \ref{eq:stable}
  \item no. of dumps: number of fiances that have been dumped during match-making, see section \ref{mods}
  \item no. of singles: number of single nodes that remain after match-making
  \item optimality index: as defined above, see section \ref{mods}
\end{itemize}

\section{Simulation Results and Discussion}

\subsection{Instabilities}

\subsection{Singles}

\subsection{Optimality Index}

\begin{figure}[t]
	\centering
	\begin{subfigure}[b]{0.4\textwidth}
	  \includegraphics[trim=0 180 0 180, clip, width=\textwidth]{../figures/figure_14}
	  \caption{bla bla bla}
	  \label{fig:optimality1}
	\end{subfigure}
  \begin{subfigure}[b]{0.4\textwidth}
	  \includegraphics[trim=0 180 0 180, clip, width=\textwidth]{../figures/figure_1}
	  \caption{bla bla bla}
	  \label{fig:optimality2}
	\end{subfigure}
\end{figure}

Question: How does the optimality index change when the visibility radius and the preference change rate are varied? Is there a dependence
on the input size?

We expect to find that that the optimality index increases for higher preference change rates.


\begin{figure}[h!]
	\includegraphics[trim=0 180 0 180, clip, width=\linewidth]{../figures/figure_4}
	\caption{bla bla bla}
	\label{fig:optimality3}
\end{figure}


\section{Summary and Outlook}

\renewcommand{\refname}{\section{References}}

\bibliographystyle{plainnat}
\bibliography{references}

% jstor pdf link http://www.jstor.org/stable/pdfplus/2312726.pdf

\section{Appendix: MATLAB Codes}

\subsection*{generateRandom.m}
\lstinputlisting{../../code/generateRandom.m}
\subsection*{generatePlane.m}
\lstinputlisting{../../code/generatePlane.m}
\subsection*{vprintf.m}
\lstinputlisting{../../code/vprintf.m}
\subsection*{makeMatch.m}
\lstinputlisting{../../code/makeMatch.m}
\subsection*{checkEngagements.m}
\lstinputlisting{../../code/checkEngagements.m}
\subsection*{simulation.m}
\lstinputlisting{../../code/simulation.m}
\subsection*{simulation2.m}
\lstinputlisting{../../code/simulation2.m}
\subsection*{simulation3.m}
\lstinputlisting{../../code/simulation3.m}
\subsection*{iteratesimulation.m}
\lstinputlisting{../../code/iteratesimulation.m}
\subsection*{plot3.m}
\lstinputlisting{../../code/plot3.m}






\end{document}  



 
